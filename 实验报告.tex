\documentclass[a4paper,9pt]{article}
\usepackage{graphicx}
\usepackage{ctex}             % 这个宏包支持中文环境
\usepackage{amsmath,amssymb,amsfonts}
\usepackage{algpseudocode}
\usepackage{algorithm}
\usepackage{float}
\usepackage{minted} % 语法高亮和代码样式设置方面更加强大和灵活
\usepackage{listings}  %引入listings包,用于在文档中插入代码,并可自定义代码样式
\usepackage[a4paper, left=2cm, right=2cm, top=2cm, bottom=2cm]{geometry} % 设置页面边距
% \setlength{\parindent}{2pt}

\begin{document}
\title{实验报告}
\author{软31 万璐瑶 2023012127}
\maketitle
\section{实验要求}
实现求平面上最近点对的复杂度为Θ(nlgn)的算法,要求:
\begin{itemize}
    \item 能够随机生成大量平面点(要求可达到一千万个点);
    \item 有方便测试的界面或者命令行,能够读入文本文件(含平面点位置信息)作为输入。
    \item 分析比较在不同输入规模情况下Θ(n2)和Θ(nlgn)算法的实际运行时间,提交实验报告。
\end{itemize}
代码运行方式以及运行参数详见readme.txt文件。
\section{实验环境}
\begin{itemize}
    \item 操作系统:Windows 11
    \item 编程语言:C++
    \item 编译器:g++
    \item IDE:Visual Studio Code
\end{itemize}

\section{算法分析}

在算法实现部分,我们实现了两种算法,分别是$O(n\lg n)$的分治算法和$O(n^2)$的暴力算法。分治算法的基本思想是将平面点集分为两部分,分别求解两部分的最近点对,然后再求解跨越两部分的最近点对,复杂度为$O(n\lg n)$。

暴力算法的基本思想是枚举遍历所有点对,找出距离最小的点对,复杂度为$O(n^2)$。

分治算法中先对输入的点对进行归并排序,保证y值升序排序,\verb|find_closest_pair2|中将点集分为两部分,分别求解两部分的最近点对,然后求解跨越两部分的最近点对。跨越两部分的最近点对的求解方法是:由于所有点已经按照y值升序排序,由课上所讲内容得知,在$\delta d$范围内,对每个点最多只需计算与其后面的7个点的距离,便可这个点与所有点的找出最小的距离,因此合并操作的复杂度为$O(n)$。

又主定理可知,分治算法的时间复杂度为$O(n\lg n)$。

\begin{listing}[H]
	\caption{找最近点对}
	\label{code:2}
	\begin{minted}{cpp} 
int find_closest_pair2(vector<pairr> pairs) { // 优化后的算法
    if (pairs.size() == 2) {
        return distance(pairs[0], pairs[1]); // 两个点直接求距离
    } else if (pairs.size() == 1) {
        return INT_MAX; // 一个点无法求最近点对距离,设为最大值
    }
    int mid = pairs.size() / 2;
    vector<pairr> left_pairs, right_pairs;
    for(int i = 0; i < mid; i++) {
        left_pairs.push_back(pairs[i]);
    }
    for(int i = mid; i < pairs.size(); i++) {
        right_pairs.push_back(pairs[i]);
    }
    int d1 = find_closest_pair2(left_pairs);
    int d2 = find_closest_pair2(right_pairs); // 分别递归求左右两边的最近点对
    // 合并两侧求最近点对
    int d = min(d1, d2);
    int minn = d;
    for(int i = 0; i < pairs.size(); i++) {
        for (int j = 1; j < 8; j++) {
            if (i + j < pairs.size()) {
                if ((pairs[i].x - pairs[i + j].x) * (pairs[i].x - pairs[i + j].x) < d) {
                    int dis = distance(pairs[i], pairs[i + j]);
                    if (dis < minn) {
                        minn = dis;
                    }
                }
            }
        }
    }
    return minn;
}
\end{minted}
\end{listing}

    
\section{实验设计}
可见分治算法的复杂度为$O(n\lg n)$,暴力算法的复杂度为$O(n^2)$,因此我们设计了两种算法,分别求解最近点对,然后比较两种算法的运行时间。再调用<chrono>库中的函数,计算两种算法的运行时间,此时`duration`的单位为纳秒,目的是更好地精确比较在数据规模较小的时候两种算法的时间差。

同时通过读取输入数的方式,设计了两种读取数据的方法,用户可通过文件输入,自带的三个.txt文件分别为100个点,500个点和1000个点,用户也可自行输入文件,文件格式为每行一个点,x和y坐标用空格隔开,且首行表示该文件内的点数。
同时,我们设计了一个随机生成平面点的函数\verb|random()|,用户也可以通过输入2选择随机生成数据,生成随机数调用的函数是\verb|rand()|,在C语言中,\verb|rand()| 函数生成的随机数范围是从 \verb|0| 到 \verb|RAND_MAX|。\verb|RAND_MAX| 是一个定义在 \verb|<stdlib.h>| 中的宏,表示 \verb|rand()| 能生成的最大值。
通常情况下,\verb|RAND_MAX| 的值为 \verb|32767|,故生成的点的x和y坐标均为0到32767之间的随机数,生成的点的数量可由用户输入。经测试,改函数生成的点的数量可达到一千万个点。

生成数据后通过命令行读取用户命令调用两种算法,便于测试时间。

\section{结果分析}
\begin{table}[!htp]
    \centering
    \caption{时间复杂度分析}
    \begin{tabular}{l|c|c}
        \hline
        n  & $O(n\lg n)$算法 & $O(n^2)$算法\\
        \hline
        100 & 0 & 0 \\
        \hline
        500 & 1556800 & 2683900 \\
        \hline
        1000 & 6508600 & 10882200 \\
        \hline
        5000 & 12133300 & 258947300 \\
        \hline
        10000 & 29822400 & 1033896700 \\
        \hline
        50000 & 161094400  & 22109884700 \\
        \hline
        100000 & 335465000 & / \\
        \hline
    \end{tabular}
\end{table}
可见复杂度为$O(n\lg n)$的算法基本随$n$的增长呈线性变化,而复杂度为$O(n^2)$的算法随$n$的增长呈指数级变化。当$n>50000$时,$O(n^2)$算法的运行时间长于一分钟,是低效的算法。

在数据较小时($n<100$),两种算法的运行时间差异不大,但随着数据量的增大,$O(n\lg n)$算法的优势逐渐显现,因此在大数据量下,$O(n\lg n)$算法是更优的选择。
\end{document}
